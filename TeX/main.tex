\documentclass{article}
\usepackage[scale=10pt]{num2path}
\usetikzlibrary{fadings}
\usepackage{listings}
\lstset{basicstyle=\small\ttfamily, numbers=left, language=C}


\begin{document}
\section{Test num2path}
A simple test:\\
xxx, 0, 1, 2, 3, 4, 5, 6, 7, 8, 9\\
{\ttfamily xxx, 0, 1, 2, 3, 4, 5, 6, 7, 8, 9}\\
\texttt{xxx}, \foreach \i in {0, ..., 9}
  {
    % \expandafter\numtopath\expandafter{\i}, 
    \numtopath{\i},
  }

\numtopath[draw=none, fill=blue]{8},
\numtopath[draw=blue, fill=red]{01},
\numtopath[draw=red, scale=2]{18},
\numtopath[draw=none, scale=2, path fading=south, fill=teal]{18},  % NOTE: fading will cost too much time !!!


\section{test lstlisting}
\let\orithelstnumber\thelstnumber
\def\thelstnumber{\numtopath[path fading=east, fill=blue]{\orithelstnumber}}

\begin{lstlisting}
#include <stdio.h>

int main() {
  printf("Hello, World!\n");
  return 0;
}
\end{lstlisting}

% \newbox\ttt
% \sbox{\ttt}{\hbox{abc def}}
% \showbox\ttt


\section{test line break}
\hskip10cm 012357325946349865819346587293648759463

\noindent\hskip10cm\numtopath[fill=teal]{012357325946349865819346587293648759463}

\section{Test font size cmd}
\texttt{108, \small 108, \Large 108, 108}

\noindent\numtopath[fill=blue]{108},
\small\numtopath[fill=purple]{108},
\Large\numtopath[fill=purple]{108},
\numtopath[draw=red, scale=2]{108}.


\makeatletter
ORI:\f@size,
\small\f@size,
\large\f@size,
\Large\f@size,
\huge\f@size,
\Huge\f@size.
\end{document}
\documentclass{l3doc}
\usepackage[scale=10pt]{num2path}
\usepackage{listings, hyperref}
\lstset{basicstyle = \footnotesize\ttfamily, numbers = left}

\usepackage[mono = false]{libertine}
\AddToHook{env/function/before}{\vspace*{-.7\baselineskip}}
\AddToHook{env/syntax/after}   {\par\vspace*{.2\baselineskip}}
\makeatletter
\def \@key  #1{\textcolor{red}{\textbf{\texttt{#1}}}\:\texttt{=}\:}
\def \s@key #1{\textcolor{red}{\textbf{\texttt{#1}}}}
\DeclareDocumentCommand \key s {\IfBooleanTF{#1}\s@key\@key}
\DeclareCommandCopy \val \meta
\def \TFF {true\textup{\textbar\textbf{false}}}
\def \TTF {\textup{\textbf{true}\textbar}false}
\def \HoLogo@ApLaTeX #1{%
  \HOLOGO@mbox {A\kern -.05em p\kern -.05em \hologo{LaTeX}}}
\newcounter{example}[subsection]
\renewcommand\theexample{\thesection.\arabic{example}}
\newwrite\example@out
\def\example@name{\jobname.example.aux}
\long\def\example@start{\begingroup\@bsphack
  \immediate\openout\example@out=\example@name
  \let\do\@makeother\dospecials\catcode`\^^M\active
  \def\verbatim@processline{\immediate\write\example@out{\the\verbatim@line}}^^A
  \verbatim@start}
\long\def\example@end{\immediate\closeout\example@out\@esphack\endgroup
  \trivlist\item[]\relax
    \leavevmode\hbox to \z@{^^A
      \hbox to \z@{\hss{\footnotesize[\theexample]}\hskip10pt}^^A
      \hskip\parindent
      \begin{minipage}[c]{.64\textwidth}^^A
        \small\verbatiminput \example@name^^A
      \end{minipage}^^A
      \fbox{^^A
        \begin{minipage}[c]{.32\textwidth}^^A
          \normalsize\input \example@name^^A
        \end{minipage}^^A
      }^^A
    \hss}^^A
  \endtrivlist}
\newenvironment{example}
  {\stepcounter{example}\example@start}{\example@end}
\makeatother
\newlist{keyval}{itemize}{10}
\setlist[keyval]{leftmargin = 0pt, labelsep = 0pt}
\makeindex

\title{^^A
  The \pkg{char2path} Bundle\thanks{%
    \url{https://github.com/zongpingding/char2path},
    \texttt{https://ctan.org/pkg/char2path}^^A
  }^^A
}
\author{^^A
  Eureka\thanks{^^A
    \href{mailto:<email>@<domain>}{\texttt{<email>@<domain>}}},~
  Mingyu Xia\thanks{^^A
    \href{mailto:xiamingyu@westlake.edu.cn}{\texttt{xiamingyu@westlake.edu.cn}}^^A
  }^^A
}
\date{Released 2025-08-01\quad \texttt{v1.0.0}}

\begin{document}

\maketitle

\begin{documentation}

\section{Introduction}

The \pkg{num2path} package (conducted with \LaTeX3) provides
a \LaTeX\ package that converts characters into Ti\textit k\/Z paths.
It supports various compilation methods, such as \hologo{pdfLaTeX},
\hologo{XeLaTeX}, \hologo{ApLaTeX} (pending for check), \hologo{LuaLaTeX}, etc.

\section{Usage}

To load this package, write the line
\begin{quote}
  |\usepackage[scale = |\meta{int}|pt]{num2path}|
\end{quote}

Here, the integral number \meta{int} supports the following choices,
and the converted paths will be scaled to the corresponding factors.

\begin{center}
  \begin{tabular}{*9c}
    \toprule
    Pound   & |10pt|  & |11pt|  & |12pt|  &
    |13pt|  & |14pt|  & |15pt|  & |16pt|  & (Pending...)\\
    \midrule
    Factor  & |0.820| & |0.902| & |0.984| &
    |1.066| & |1.148| & |1.230| & |1.311| & ...\\
    \bottomrule
  \end{tabular}
\end{center}

\begin{function}{\numtopath}
  \begin{syntax}
    \cs{numtopath} \oarg{key-vals} \marg{string}
  \end{syntax}
  The mandatory argument accepts the required input string, and the optional
  argument accepts the following keys to set the style how the string converts
  to path.
  \begin{keyval}
    \item [\key{draw,fill}] \val{color} can set the color of the outline/fill of string
    (Default: |black|).
    \item [\key{scale}] \val{fp num} can set the scale of the string
    (Default: |1|).
    \item [\key{path fading}] \val{direction} can set the direction of
    fading of the string.
  \end{keyval}
\end{function}

\begin{function}{\numfading}
  \begin{syntax}
    \cs{numtopath} \oarg{key-vals} \marg{string}
  \end{syntax}
  The mandatory argument accepts the required input string, and the optional
  argument accepts the following keys to set the style how the string converts
  to path.
  \begin{keyval}
    \item [\key{colorset}] \val{clist} can set the two colors for fading
    \item [\key{direction}] \val{direction} can set the direction of fading
  \end{keyval}
\end{function}

\subsection{Basic usage}

\noindent
The basic usage only print the outline of the inputted string.
\begin{example}
 \numtopath{0123456789}
\end{example}
\noindent
Users can pecific the color of outline and fill,
or specific the scale of inputted characters.
Just like the keys in \pkg{Ti\textit k\/Z}
\begin{example}
 \numtopath[draw = none, fill = blue]{0123456789}
\end{example}

\begin{example}
 \numtopath[draw = blue, fill = red]{0123456789}
\end{example}

\begin{example}
 \numtopath[draw = red, scale = 2]{0123456789}
\end{example}

Fading is also supported

\begin{example}
 \numtopath[ draw = none, path fading = south,
             scale = 2, fill = teal ] {0123456789}
\end{example}
\begin{texnote}
  Fading will cost too much time !!!
\end{texnote}

\subsection{Use in \pkg{listings}}

This package can be used in the \pkg{listings} package, so that users could
directly copy and paste the source code from |.pdf| file without worry about
including unwanted line numbers.

\begin{verbatim}
  \let\orithelstnumber\thelstnumber
  \def\thelstnumber{\numtopath[path fading = east, fill=blue]{\orithelstnumber}}
\end{verbatim}
\let\orithelstnumber\thelstnumber
\def\thelstnumber{\numtopath[path fading = east, fill=blue]{\orithelstnumber}}

\begin{example}
 \begin{lstlisting}[language = C, gobble = 4]
    #include <stdio.h>
    
    int main() {
      printf("Hello, World!\n");
      return 0;
    }
 \end{lstlisting}
\end{example}

% \newbox\ttt
% \sbox{\ttt}{\hbox{abc def}}
% \showbox\ttt

\hskip1em 0123456789

\noindent\hskip2em
\numtopath[fill=teal]{0123456789}

\subsection{Use in text}

\texttt{0123456789, \small 0123456789, \Large 0123456789}

\noindent

\numtopath[fill=blue]{0123456789},
\small \numtopath[fill=purple]{0123456789},
\Large \numtopath[fill=purple]{0123456789}.

\makeatletter
ORI:\f@size,
\small\f@size,
\large\f@size,
\Large\f@size,
\huge\f@size,
\Huge\f@size.

\normalfont \normalsize

\subsection{Fading string}

If the users want to fading the whole strings, not fading one by one,
then they should use this module%
\footnote{Developed by \href{github.com/myhsia}{Mingyu Xia}}.

\subsubsection{Discretely fading string}

\numtopath[path fading = east, fill = red]{1234567890}

\subsubsection{Continously fading string}

\indent

\begin{example}
 \numfading[colorset = {red, green},
            direction = horizontal] {1234567890}
\end{example}

\begin{example}
 \numfading[colorset = {violet, blue},
            direction = horizontal] {1234567890}
\end{example}

\begin{example}
 \numfading[colorset = {green, red},
            direction = horizontal] {1234567890}
\end{example}

\begin{example}
 \numfading[colorset = {blue, violet},
            direction = horizontal] {1234567890}
\end{example}

% \numfading[colorset = teal, direction = south]{20250721}

% \numfading[colorset = violet, direction = north]{20250721}

\end{documentation}

\PrintIndex

\section{Todo list}

\begin{itemize}
  \item Merge the |Fading string| module (commmand) into \cs{numtopath}
  \item ...
\end{itemize}

\end{document}